% SYLLABUS ----
% This is just here so I know exactly what I'm looking at in Rstudio when messing with stuff.
\documentclass[11pt,]{article}
\usepackage[margin=1in]{geometry}
\newcommand*{\authorfont}{\fontfamily{phv}\selectfont}
\usepackage[]{mathpazo}
\usepackage{abstract}
\renewcommand{\abstractname}{}    % clear the title
\renewcommand{\absnamepos}{empty} % originally center
\newcommand{\blankline}{\quad\pagebreak[2]}

\providecommand{\tightlist}{%
  \setlength{\itemsep}{0pt}\setlength{\parskip}{0pt}}
\usepackage{longtable,booktabs}

\usepackage{parskip}
\usepackage{titlesec}
\titlespacing\section{0pt}{12pt plus 4pt minus 2pt}{6pt plus 2pt minus 2pt}
\titlespacing\subsection{0pt}{12pt plus 4pt minus 2pt}{6pt plus 2pt minus 2pt}

\titleformat*{\subsubsection}{\normalsize\itshape}

\usepackage{titling}
\setlength{\droptitle}{-.25cm}

%\setlength{\parindent}{0pt}
%\setlength{\parskip}{6pt plus 2pt minus 1pt}
%\setlength{\emergencystretch}{3em}  % prevent overfull lines

\usepackage[T1]{fontenc}
\usepackage[utf8]{inputenc}

\usepackage{fancyhdr}
\pagestyle{fancy}
\usepackage{lastpage}
\renewcommand{\headrulewidth}{0.3pt}
\renewcommand{\footrulewidth}{0.0pt}

\lhead{}
\chead{}
\rhead{\footnotesize CJ 4310: Special Problems in the Criminal Justice
System -- Fall 2025}
\lfoot{}
\cfoot{\small \thepage/\pageref*{LastPage}}
\rfoot{}

\fancypagestyle{firststyle}
{
\renewcommand{\headrulewidth}{0pt}%
   \fancyhf{}
   \fancyfoot[C]{\small \thepage/\pageref*{LastPage}}
}

%\def\labelitemi{--}
%\usepackage{enumitem}
%\setitemize[0]{leftmargin=25pt}
%\setenumerate[0]{leftmargin=25pt}




\makeatletter
\@ifpackageloaded{hyperref}{}{%
\ifxetex
  \usepackage[setpagesize=false, % page size defined by xetex
              unicode=false, % unicode breaks when used with xetex
              xetex]{hyperref}
\else
  \usepackage[unicode=true]{hyperref}
\fi
}
\@ifpackageloaded{color}{
    \PassOptionsToPackage{usenames,dvipsnames}{color}
}{%
    \usepackage[usenames,dvipsnames]{color}
}
\makeatother
\hypersetup{breaklinks=true,
            bookmarks=true,
            pdfauthor={ ()},
             pdfkeywords = {},
            pdftitle={CJ 4310: Special Problems in the Criminal Justice
System},
            colorlinks=true,
            citecolor=blue,
            urlcolor=txst,
            linkcolor=txst,
            pdfborder={0 0 0}}
\urlstyle{same}  % don't use monospace font for urls


\setcounter{secnumdepth}{0}





\usepackage{setspace}

\title{CJ 4310: Special Problems in the Criminal Justice System}
\author{Seth Watts, PhD}
\date{Fall 2025}

\usepackage{tikz}

\newcommand{\shrug}[1][]{%
\begin{tikzpicture}[baseline,x=0.8\ht\strutbox,y=0.8\ht\strutbox,line width=0.125ex,#1]
\def\arm{(-2.5,0.95) to (-2,0.95) (-1.9,1) to (-1.5,0) (-1.35,0) to (-0.8,0)};
\draw \arm;
\draw[xscale=-1] \arm;
\def\headpart{(0.6,0) arc[start angle=-40, end angle=40,x radius=0.6,y radius=0.8]};
\draw \headpart;
\draw[xscale=-1] \headpart;
\def\eye{(-0.075,0.15) .. controls (0.02,0) .. (0.075,-0.15)};
\draw[shift={(-0.3,0.8)}] \eye;
\draw[shift={(0,0.85)}] \eye;
% draw mouth
\draw (-0.1,0.2) to [out=15,in=-100] (0.4,0.95);
\end{tikzpicture}}

\linespread{1.05}
\usepackage{xcolor}
\definecolor{txst}{HTML}{501214}

\begin{document}

		\maketitle
	

		\thispagestyle{firststyle}

%	\thispagestyle{empty}


	\noindent \begin{tabular*}{\textwidth}{ @{\extracolsep{\fill}} lr @{\extracolsep{\fill}}}


E-mail: \texttt{\href{mailto:sbwatts@txstate.edu}{\nolinkurl{sbwatts@txstate.edu}}} & Web: \href{http://sethbwatts.com/courses}{\tt sethbwatts.com/courses}\\
Office Hours: By appt. via Zoom  &  Class Hours: Asynchronous\\
Office: Hines 112  & Class Room: Zoom\\
	&  \\

	\hline
	\end{tabular*}

\vspace{2mm}



\section{Overview}\label{overview}

\textbf{Course Catalog Description}: This course is a study of
contemporary problems in the administration, management, organization
and operation of criminal justice agencies.

\textbf{Prerequisites}: CJ 2310 and CJ 2350 and CJ 2355 and CJ 2360 all
with grades of ``D'' or better. Corequisite: CJ 3346 with a grade of
``D'' or better.

\textbf{Required Textbook}:
\href{https://www.homeworkforyou.com/static_media/uploadedfiles/1714949674_779199__116..pdf}{Mays,
G. L., \& Ruddell, R. (2019). Making sense of criminal justice: Policies
and practices. Oxford University Press.} - No cost, PDF available.

\textbf{Additional Readings/ Resources}:

\begin{itemize}
\item
  Required articles will be provided by the instructor in PDF format on
  Canvas.
\item
  Additional videos will be provided by the instructor on Canvas.
\item
  Access to reliable Internet, Canvas, and Microsoft Word/ some other
  word processing software.
\end{itemize}

\textbf{Course Rationale}: This class will broadly cover crime control
strategies. The primary focus will be on how the Criminal Justice System
attempts to control criminal activity. Specifically, we will look at the
role of the police, courts, and corrections in reducing offending. We
will also cover how the community, environment, labor market,
technological innovations, and legislation can be utilized to control
crime.

\textbf{Course Objectives}:

\begin{itemize}
\item
  Understand important contemporary issues and problems concerning crime
  and criminal justice.
\item
  Be able to critically examine criminal justice policies and their
  evidence base.
\item
  Ability to distinguish between evidence-based and non-evidence-based
  policies
\item
  Synthesize research materials and integrate what they have learned
  into undergraduate-level writing.
\end{itemize}

\textbf{Format of Class}: This class is entirely asynchronous online.
Students are responsible for reading the text materials and other
required materials.

\section{Assignments and Evaluation}\label{assignments-and-evaluation}

\subsection{Summary of Assignments}\label{summary-of-assignments}

Below are the assignments and their points for the class:

\begin{table}[!h]
\centering
\resizebox{\ifdim\width>\linewidth\linewidth\else\width\fi}{!}{
\begin{tabular}{lll}
\toprule
Assignment & Points & Percentage\\
\midrule
\cellcolor{gray!10}{Tweet Summaries} & \cellcolor{gray!10}{6.25 per (x12) = 75} & \cellcolor{gray!10}{11.5\%}\\
Discussion Boards & 20 per (3x) = 60 & 9.2\%\\
\cellcolor{gray!10}{Paper Topic} & \cellcolor{gray!10}{40} & \cellcolor{gray!10}{6.2\%}\\
Paper Outline & 75 & 11.5\%\\
\cellcolor{gray!10}{Final Paper} & \cellcolor{gray!10}{100} & \cellcolor{gray!10}{15.4\%}\\
\addlinespace
Midterm Exam & 150 & 23.1\%\\
\cellcolor{gray!10}{Final Exam} & \cellcolor{gray!10}{150} & \cellcolor{gray!10}{23.1\%}\\
Total & 650 & 100\%\\
\bottomrule
\end{tabular}}
\end{table}

\textbf{\emph{Discussion Boards}}: There will be three total discussion
boards throughout the course. You will be responsible for a discussion
board post and peer response worth 10 points each. The initial post
should be at least 250 words in length and adequately addresses each
part of the given prompt. Second, there should be two classmate response
posts that should be at least 100 words in length and related to the
substance of the classmate's post. These will be due Sunday by midnight.

\textbf{\emph{Assigned Readings}}: Unless otherwise specifically
assigned, reading assignments are taken from the required resources
specified in this syllabus. The reading assignments listed for each
class are those readings that a student must complete before reviewing
the corresponding module's lecture or completing corresponding module
assignments. Not all of the assigned readings will be discussed in
lectures; however, all assigned readings may be covered on exams. See
the course schedule for reading assignments.

\textbf{\emph{Tweet Summaries}}: Each week you will submit a brief
summary -no more than 280 words- to me via Canvas (Yes, this is the max
limit of characters on X). This will be a summary of the readings for
that particular week. The key point here is to be concise. Yes, this is
a challenge but it is important to be able to synthesize research and
disseminate it in ways that can be easily digested by readers. This
assignment will help you think critically about the content, what should
be included, and how to perfect your virtual ``elevator pitch.''

\textbf{\emph{Paper Topic}}: In preparation for the paper outline and
final paper, you will submit a paper topic to me via Canvas. The paper
topic should be a criminal justice policy or topical area that we cover
in class. You will indicate whether you plan to write a policy paper or
literature review (see below for details). Ideally this will be a short
and sweet one sentence paper topic but it can be two or three sentences.
You can look ahead in the course readings to generate a paper topic. It
is entirely up to you but must be within the confounds of the course
topics. If it is a related topic but not covered in the class, let me
know and we can discuss.

\textbf{\emph{Paper Outline}}: To help facilitate the development of
your final paper, you will submit a paper outline via Canvas. This
outline should be no more than three pages. It should include section
and subsection headers, potential evidence to bring in, citations, and
arguments to be made. It should serve as the scaffolding for your final
paper. This is not a set in stone outline, if you deviate, you will not
be penalized. However, if you change your paper topic after the outline
is submitted, please notify me. Be sure to include a reference page (not
included in the approximately 3 page limit).

\textbf{\emph{Final Paper}}: You have two options for your final paper:
a \textbf{Policy paper} or \textbf{Literature review}.

The final paper should be approximately 8-10 pages (double spaced), with
5 different references from the class readings and 5 external references
that were not covered in class. You should be using APA formatting for
all of your writing assignments.\footnote{See the following link for a
  guide to APA formatting:
  \url{https://owl.purdue.edu/owl/research_and_citation/apa_style/apa_formatting_and_style_guide/index.html}}
Be sure to include a reference page (not included in the 8-10 page
requirement).

\begin{itemize}
\item
  \textbf{\emph{Policy paper}}: This is narrowly tailored to a
  practitioner audience such as the local municipal police department,
  city counsel, local community groups, etc. You should identify a
  problem (issue covered in this course), and provide a well structured
  paper that highlights an evidence-based policy that will (or will not)
  address the issue you identified. You should be sure to provide
  information related to the issue and evidence for the policy you are
  discussing. This should be a hybrid paper between expository and
  persuasive. You want to provide the audience with ample information
  about the policy (the good and bad) but if the policy is
  evidence-based, you want to highlight this fact and suggest that the
  policy be adopted. If the policy is not evidence-based, you should
  explain why and that it is best not to be adopted.
\item
  \textbf{\emph{Literature review}}: Broad in scope and intended for an
  academic audience. This should be a paper that synthesizes the
  research of your chosen topic, providing the background, examples,
  strengths and weaknesses, and a conclusion on the efficacy of your
  chosen policy/intervention. this paper should be structured as an
  expository article.
\end{itemize}

\textbf{\emph{Midterm Exam}}: Your midterm exam will consist of 75
multiple choice questions. The midterm will cover the first 7 weeks of
content. These will be online in Canvas and will be due Friday night by
midnight. Once the exam is accessed you will have 1 hour and 30 minutes
to complete the exam.

\textbf{\emph{Final Exam}}: Your final exam will consist of 75 multiple
choice questions. The final exam will cover the final 6 weeks of
content. These will be online in Canvas and will be due Friday night by
midnight. Once the exam is accessed you will have 1 hour and 30 minutes
to complete the exam.

\section{Course Policies}\label{course-policies}

\textbf{Calendar and Due Dates}: In addition to the information located
in each week's module in Canvas, a master calendar of due dates will be
made available under the `Calendar' section of the course homepage in
Canvas. Should any adjustments be needed to original due dates, I will
post an announcement to the `Announcements' section and will adjust the
calendar to reflect the modified timeline. This is NOT a self-paced or
correspondence course.

\textbf{Late Work}: Written assignments are to be submitted by the due
date by the times specified (Central Time). In extreme cases, as
approved by the instructor, late work will be accepted by an agreed upon
alternate due date. Work submitted late for other reasons will be
penalized 10\% for each day late. Notify the instructor BEFORE an
assignment is due if an urgent situation arises and the assignment will
not be submitted on time. Published assignment due dates are firm.

\textbf{Grade Records}: To the extent possible, I will try to make sure
that grades for the assignments are posted within approximately three
days after its due date. On occasion, some assignments require
intermediate feedback in which a longer turnaround should be expected.
Once the assignment is evaluated, its grades will be placed into the
Canvas gradebook and will be available to you. Exam/quiz grades will be
posted upon completion of the exam/quiz. Individual reports will not be
sent apart from the information recorded in Canvas, so you should
periodically monitor the gradebook to assure your assignments have been
received and graded.

\textbf{Academic Dishonesty}: All students taking classes in Criminal
Justice must subscribe to the Texas State University Honor Code
(\url{http://www.txstate.edu/honorcodecouncil/Academic-Integrity.html})
and Code of Student Conduct
(\url{http://www.dos.txstate.edu/handbook/rules/cosc.html}). Failure to
adhere to any component of these documents may entail consequences
ranging from serious (e.g.~unexcused absences, zero points assigned for
exam grades, etc.) to severe (a course grade of ``F'' or even dismissal
from the University).

\textbf{Artificial Intelligence}: In this course, you are welcome and
encouraged to use artificial intelligence platforms such as
\textbf{ChatGPT} ONLY for pre-writing, brainstorming, and locating
sources, unless otherwise specified in assignment instructions. You
should not use AI to produce your own assignments or otherwise perform
tasks that you are expected to be able to do or learn to do on your own.
In short, ChatGPT isn't taking this course; you are. You are here to
learn, not to cut corners. Please take that responsibility seriously.

Your instructor will make every effort to be transparent in describing
why you should complete each assignment, what skills you will develop,
and when/ how it is appropriate to use AI as a tool for completing the
assignments.

\textbf{Avoiding Plagiarism}: Some students truly do not understand what
plagiarism is, and therefore plagiarize unwittingly or unintentionally.
But ignorance is not an excuse for unethical academic conduct.

Moreover, read the following rules that apply regardless of the citation
form or style you may be using.

\begin{enumerate}
\def\labelenumi{\arabic{enumi}.}
\item
  \emph{Direct Quotations} -- Whenever you directly quote someone else,
  you must provide a citation to the source of the material from which
  you are quoting. Moreover, you must put the material in quotation
  marks or otherwise set it off in an indented quote so the reader knows
  what words are yours and what words are quoted. It is unacceptable to
  use the words of others and only partially quote the original source.
  This is true even if you provide citation to the source both in text
  and in your references section!
\item
  \emph{Paraphrasing/Indirect Quotations} -- Whenever you indirectly
  quote someone else (i.e., you paraphrase the work of another), you
  must provide a citation to the source of the material from which you
  are paraphrasing. Simply changing the structure of a sentence, or a
  few words in a sentence so that the sentence you write is not an exact
  quote from the original source does not mean a citation is not needed.
  This is because the idea you are expressing is not your own, but
  rather someone else's.
\item
  \emph{Using Other's Ideas} -- Even if you compose an entire paragraph
  of writing in your own words (i.e., neither quoted nor paraphrased),
  if the idea you are expressing in that paragraph is not your own,
  original idea, you must provide a citation to the source from which
  you obtained this idea.
\item
  \emph{Collaborative Work} -- If you collaborate on any work with
  someone else and fail to acknowledge that collaboration, you are
  guilty of plagiarism. If you have received permission from you
  professor to collaborate on some assignment, be sure that all of the
  contributor's names appear on the submission.
\item
  \emph{Altering or Revising Another's Work} -- If you alter or revise
  the work done by someone and submit that work as your own, you have
  plagiarized. Similarly, if you allow someone else to alter or revise
  work that you have done and then allow that person to submit it as his
  or her own work, you are both guilty of plagiarism. Work that is not
  entirely your own must be credited by citation, both in text and in
  your references page.
\item
  \emph{Altering or Revising Your Own Prior Work} -- You should also be
  aware that reusing or revising your own work that was prepared for
  another class or another professor, and not bringing it to the
  attention of the professor to whom you are submitting the revised work
  is also academic dishonesty. If, for example, you have two classes
  that require a term paper, and you can write one paper that meets the
  requirements of both classes, you may not submit that paper to both
  professors unless you get permission to do so in advance from both
  professors. Similarly, if you wrote a paper several semesters ago that
  can be revised and submitted in satisfaction of a paper requirement
  for a course in which you are currently enrolled, doing so is academic
  dishonesty unless you get the advanced permission of your professor to
  do so. The reason this is dishonest is that it is not an original work
  prepared in satisfaction for the requirements on the course you are
  currently taking. Contact your instructor for written approval if you
  are seeking an exception for unique cases.
\end{enumerate}

\textbf{Canvas}: Along with this syllabus, all grades in this course
will be posted to CANVAS. I will do my best to post the grades within a
week of an exam. To access CANVAS, you can follow the following link:
\url{https://canvas.txstate.edu}. Should you need a laptop, the library
loans them out to students. For more information, please contact the
library or go to:
\url{http://www.library.txstate.edu/about/departments/circ/laptop}.

\textbf{Disability}: Students with a documented disability and/or those
that believe they have a legitimate disability, are strongly encouraged
to arrange a meeting with me to ensure they receive appropriate
accommodations as set forth by the Office of Disabled Students. Students
must contact the instructor within the first two weeks of the semester.
If accommodations are needed, please contact the Office of Disabilities
Services, as follows:

\begin{itemize}
\item
  Office of Disability Services Suite 5-5.1, LBJ Student Center 601
  University Drive San Marcos TX 78666 Phone: 512.245.3451 Fax:
  512.245.3452 Office
\item
  Hours: Monday to Friday 8am -5pm Website:
  \url{http://www.ods.txstate.edu/}
\end{itemize}

\pagebreak

\section{Course Schedule}\label{course-schedule}

\begin{table}[!h]
\centering
\resizebox{\ifdim\width>\linewidth\linewidth\else\width\fi}{!}{
\begin{tabular}{rl>{\raggedright\arraybackslash}p{7cm}>{\raggedright\arraybackslash}p{7cm}>{\raggedright\arraybackslash}p{4cm}>{\raggedright\arraybackslash}p{4cm}}
\toprule
Week & Date & Topic & Readings & Media & Assignments\\
\midrule
\cellcolor{gray!10}{1} & \cellcolor{gray!10}{Aug 25-31} & \cellcolor{gray!10}{Politics and Criminal Justice Policy} & \cellcolor{gray!10}{Syllabus; Textbook: Ch. 1, 2, \& 3} & \cellcolor{gray!10}{} & \cellcolor{gray!10}{}\\
2 & Sep 1-7 & Evidence-Based Policy & Welsh \& Farrington (2011) &  & Summary 1\\
\cellcolor{gray!10}{3} & \cellcolor{gray!10}{Sep 8-14} & \cellcolor{gray!10}{Police I: Crime Control Agents} & \cellcolor{gray!10}{Kelling \& Moore (1988); Chalfin (2022)} & \cellcolor{gray!10}{} & \cellcolor{gray!10}{Summary 2}\\
4 & Sep 15-21 & Police II: Hot Spots \& Problem Oriented Policing & Braga et al. (2019); Hinkle et al. (2020) & Reducing Crime Podcast Ep. \#66 (David Weisburd) & Summary 3\\
\cellcolor{gray!10}{5} & \cellcolor{gray!10}{Sep 22-28} & \cellcolor{gray!10}{Police III: Focused Deterrence} & \cellcolor{gray!10}{Braga et al. (2018)} & \cellcolor{gray!10}{Reducing Crime Podcast Ep. \#18 (Thomas Abt)} & \cellcolor{gray!10}{Summary 4; Discussion Board \#1}\\
\addlinespace
6 & Sep 29-Oct 5 & Courts: Prosecutors \& Sentencing & Textbook: Ch. 7 &  & Summary 5\\
\cellcolor{gray!10}{7} & \cellcolor{gray!10}{Oct 6-12} & \cellcolor{gray!10}{Corrections: Prisons \& Community Supervision} & \cellcolor{gray!10}{Textbook: Ch. 11, 12, \& 13} & \cellcolor{gray!10}{} & \cellcolor{gray!10}{Summary 6; Discussion Board \#2}\\
8 & Oct 13-19 & Midterm Exam &  &  & Midterm Exam; Paper Topic due\\
\cellcolor{gray!10}{9} & \cellcolor{gray!10}{Oct 20-26} & \cellcolor{gray!10}{Community-based I: CPTED} & \cellcolor{gray!10}{Welsh \& Farrington (2008); Cozens \& Love (2015)} & \cellcolor{gray!10}{} & \cellcolor{gray!10}{Summary 7}\\
10 & Oct 27-Nov 2 & Community-based II: Public Health Approaches & Butts et al. (2015); Cornell University et al. (2023) &  & Summary 8\\
\addlinespace
\cellcolor{gray!10}{11} & \cellcolor{gray!10}{Nov 3-9} & \cellcolor{gray!10}{Technology \& Surveilance} & \cellcolor{gray!10}{Connealy et al. (2024), Piza et al. (2024)} & \cellcolor{gray!10}{} & \cellcolor{gray!10}{Summary 9; Paper Outline due}\\
12 & Nov 10-16 & Employment and Crime & Savolainen et al. (2019); Ludwig \& Schnepel (2024) & Probable Causation Podcast: Sara Heller & Summary 10; Discussion Board \#3\\
\cellcolor{gray!10}{13} & \cellcolor{gray!10}{Nov 17-23} & \cellcolor{gray!10}{Guns, Policy, \& Crime} & \cellcolor{gray!10}{Smart et al. (2023) RAND Summary; Cook (2018)} & \cellcolor{gray!10}{} & \cellcolor{gray!10}{Summary 11}\\
14 & Nov 24-Nov 30 & AI, Criminal Justice, \& Crime & Johnson et al. (2024); Ezeh et al. (2025) & Police In-Service Podcast, Episode \#7: Artifical Intelligence in Policing & Summary 12\\
\cellcolor{gray!10}{15} & \cellcolor{gray!10}{Dec 1-7} & \cellcolor{gray!10}{Final Exam} & \cellcolor{gray!10}{} & \cellcolor{gray!10}{} & \cellcolor{gray!10}{Final Exam; Final Paper due}\\
\bottomrule
\end{tabular}}
\end{table}




\end{document}

\makeatletter
\def\@maketitle{%
  \newpage
%  \null
%  \vskip 2em%
%  \begin{center}%
  \let \footnote \thanks
    {\fontsize{18}{20}\selectfont\raggedright  \setlength{\parindent}{0pt} \@title \par}%
}
%\fi
\makeatother
